%%%%%%%%%%%%%%%%%%%%%%%%%%%%%%%%%%%%%%%%%%%%%%%%%%%%%%%%%%%%%%%%%%%%%%%%%%%%%%%
% Chapter 6: Presupuesto
%%%%%%%%%%%%%%%%%%%%%%%%%%%%%%%%%%%%%%%%%%%%%%%%%%%%%%%%%%%%%%%%%%%%%%%%%%%%%%%

%++++++++++++++++++++++++++++++++++++++++++++++++++++++++++++++++++++++++++++++

%------------------------------------------------------------------------------

El presupuesto elaborado para este proyecto contiene dos partes fundamentales. Por una parte tenemos los costes tecnológicos aparejados al desarrollo y por otra debemos tener en cuenta los recursos humanos.

En primer lugar se expondrán los costes tecnológicos:

%--------------------------------------------------------------------------
\begin{table}[h]
\centering
\resizebox{\columnwidth}{!}{%
\begin{tabular}{ll|l|}
\hline
\multicolumn{1}{|l|}{Tipo}          & Descripción                                                                     & Coste         \\ \hline
\multicolumn{1}{|l|}{NPM}           & Hosting privado en NPM con múltiples paquetes                                   & \EUR{21} (\EUR{7}/mes) \\ \hline
\multicolumn{1}{|l|}{CircleCI}      & Serivicio premium de CircleCI para múltiples tests en paralelo                  & \EUR{150}          \\ \hline
\multicolumn{1}{|l|}{dominio}       & Registrar el dominio genetics.js para su utilización en la web de documentación & \EUR{50}           \\ \hline
\multicolumn{1}{|l|}{Hosting}       & Hosting privado para alojar la documentación                                    & \EUR{100}          \\ \hline
\multicolumn{1}{|l|}{Plantilla web} & Adquisición de una plantilla web para la documentación                          & \EUR{50}           \\ \hline
                                    &                                                                                 & \EUR{371}          \\ \cline{3-3} 
\end{tabular}%
}
\caption{Costes tecnológicos}
\end{table}

Para calcular los costes de recursos humanos, se debe aclarar que la duración de este proyecto es de \textbf{300} horas puesto que la asignatura \textbf{Trabajo fin de grado} cuenta con 12 créditos ECTS y cada crédito equivale a 25 horas.

%--------------------------------------------------------------------------
\begin{table}[h]
\centering
\resizebox{\columnwidth}{!}{%
\begin{tabular}{|l|l|l|}
\hline
Tipo                  & Descripción                                                        & Coste                       \\ \hline
Trabajo de desarrollo & Este trabajo tiene un coste de \EUR{15} por hora. & \EUR{4500} \\ \hline
\end{tabular}%
}
\caption{Costes de recursos humanos}
\end{table}

Finalmente, calcularemos los costes totales:

%--------------------------------------------------------------------------
\begin{table}[h]
\centering
\begin{tabular}{l|l|}
\hline
\multicolumn{1}{|l|}{Tipo}              & Costes \\ \hline
\multicolumn{1}{|l|}{Coste tecnológico} & \EUR{371}    \\ \hline
\multicolumn{1}{|l|}{Recursos humanos}  & \EUR{4500}   \\ \hline
                                        & \EUR{4871}   \\ \cline{2-2} 
\end{tabular}
\caption{Costes totales}
\end{table}

